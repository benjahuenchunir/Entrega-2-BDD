\documentclass{article}

\usepackage{hyperref}

\title{Entrega 2}
\author{Grupo 131}
\date{\today}

\begin{document}

\maketitle

\section{Introduction}

\section{Diagrama E/R}

\section{Esquema relacional}

\section{Dependencias funcionales y justificación}
\begin{enumerate}
    \item proovedores\_peliculas \\
    Las dependencias de esta tabla están en BCNF dado que no existen repeticiones de datos y cada columna depende unicamente de la llave mínimal de la siguiente forma: \\
    id\_proovedor, id\_pelicula $\rightarrow$ precio, dias\_arriendo
    \item Proveedores \\
    Las dependencias de esta tabla están en BCNF dado que no existen repeticiones de datos y cada columna depende unicamente de la llave mínimal de la siguiente forma: \\
    id $\rightarrow$ nombre, costo
    \item proveedores\_series \\
    Las dependencias de esta tabla están en BCNF dado que no existen repeticiones de datos y cada columna es independiente, por lo que no existen dependencias que transgredan BCNF. \\
    \item arriendos\_peliculas \\
    Las dependencias de esta tabla están en BCNF dado que no existen repeticiones de datos y cada columna depende unicamente de la llave mínimal de la siguiente forma: \\
    id $\rightarrow$ id\_usuario, id\_pelicula, fecha, monto, dias\_arriendo \\
    \item Subscripciones \\
    Las dependencias de esta tabla están en BCNF dado que no existen repeticiones de datos y cada columna depende unicamente de la llave mínimal de la siguiente forma: \\
    id $\rightarrow$ id\_usuario, id\_proveedor, estado, fecha\_inicio, fecha\_termino, costo \\
    \item Usuarios \\
    Las dependencias de esta tabla están en BCNF dado que no existen repeticiones de datos y cada columna depende unicamente de la llave mínimal de la siguiente forma: \\
    id $\rightarrow$ nombre, mail, password, username \\
    \item Pagos \\
    Las dependencias de esta tabla están en BCNF dado que no existen repeticiones de datos y cada columna depende unicamente de la llave mínimal de la siguiente forma: \\
    id $\rightarrow$ id\_subscripcion, fecha \\
    \item peliculas \\
    Las dependencias de esta tabla están en BCNF dado que no existen repeticiones de datos y cada columna depende unicamente de la llave mínimal de la siguiente forma: \\
    id $\rightarrow$ titulo, puntuacion, clasificacion, año, duracion \\
    \item historial\_peliculas \\
    Las dependencias de esta tabla están en BCNF dado que no existen repeticiones de datos y cada columna depende unicamente de la llave mínimal de la siguiente forma: \\
    id $\rightarrow$ id\_usuario, id\_pelicula, fecha \\
    \item historial\_series \\
    Las dependencias de esta tabla están en BCNF dado que no existen repeticiones de datos y cada columna depende unicamente de la llave mínimal de la siguiente forma: \\
    id $\rightarrow$ id\_usuario, id\_capitulo, fecha \\
    \item capitulos \\
    Las dependencias de esta tabla están en BCNF dado que no existen repeticiones de datos y cada columna depende unicamente de la llave mínimal de la siguiente forma: \\
    id $\rightarrow$ id\_serie, titulo, duracion, numero\_temporada \\
    \item series \\
    Las dependencias de esta tabla están en BCNF dado que no existen repeticiones de datos y cada columna depende unicamente de la llave mínimal de la siguiente forma: \\
    id $\rightarrow$ titulo, puntuacion, clasificacion, año \\
    \item generos\_peliculas \\
    Las dependencias de esta tabla están en BCNF dado que no existen repeticiones de datos y cada columna es independiente, por lo que no existen dependencias que transgredan BCNF. \\
     \item generos \\
    Las dependencias de esta tabla están en BCNF dado que no existen repeticiones de datos y cada columna depende unicamente de la llave mínimal de la siguiente forma: \\
    id $\rightarrow$ nombre \\
    \item generos\_series \\
    Las dependencias de esta tabla están en BCNF dado que no existen repeticiones de datos y cada columna es independiente, por lo que no existen dependencias que transgredan BCNF. \\
    \item generos\_subgeneros\\
    Las dependencias de esta tabla están en BCNF dado que no existen repeticiones de datos y cada columna es independiente, por lo que no existen dependencias que transgredan BCNF. \\
\end{enumerate}

\section{Consultas}
\begin{enumerate}
    \item SELECT peliculas.titulo AS pelicula, proveedores.nombre AS proovedor FROM peliculas
    INNER JOIN proveedores\_peliculas ON peliculas.id = proveedores\_peliculas.id\_pelicula
    INNER JOIN proveedores ON proveedores\_peliculas.id\_proveedor = proveedores.id
    WHERE proveedores\_peliculas.precio IS NULL;
    \item
\end{enumerate}

\section{Supuestos}
\begin{enumerate}
    \item Para la consulta 6, "Dado un username ingresado por el usuario, muestre todas las series para las cuales el usuario ha visto más de un capítulo durante el último año", el ver dos veces el mismo capítulo también cuenta (issue \href{https://github.com/IIC2413/Syllabus-2023-2/issues/172}{\#72}).
    \item Para las consultas que piden "en el ultimo año" se consideró un intervalo de tiempo mayor (3 años), ya que no hay datos del último año (issue \href{https://github.com/IIC2413/Syllabus-2023-2/issues/175}{\#175})
    \item Clasificación, puntuación y año son de la serie y no varían según capítulo (issue \href{https://github.com/IIC2413/Syllabus-2023-2/issues/171}{\#171})
    \item La serie "Rick  y Morty" no tiene género en los datos entregados, se le asignó "Comedia" (issue \href{https://github.com/IIC2413/Syllabus-2023-2/issues/189}{\#189})
    \item Se eliminó el atributo "estado" de la tabla de subscripciones, ya que solo representa si la subscripción esta activa o no, y eso se puede inferir de la fecha de termino (si existe o es nula). Así se evita una dependencia transitiva.
    \item Al igual que el costo de subscripción y el costo de arriendo de una película, los días de arriendo de la película (el atributo "disponibilidad") también pueden variar en el tiempo, requiriendo ser guardados para cada arriendo.
\end{enumerate}


\end{document}
